\chapter*{Preface}

% \section*{Setting}

This story is a frivolous attempt at imagining what it would be like if an archetypal Evil Overlord operating in a fantasy world setting did actually read the famous "Evil Overlord List"\footnote{\url{https://en.wikipedia.org/wiki/Evil_Overlord_List}} or EOL, took it most seriously and acted accordingly.

I see at least three problems with that premise. First problem is that EOL is a joke itself. It was never meant to be anything more than vague collection of clich\`es and tropes so often misused in the works of fantasy and adventure action genres. It does not represent a solid structured system that can be turned into a consistent policy. Writing convincing villain is much harder than convincing hero, because, at the very least, villain has to present credible threat but eventually lose.

The second problem is that the author cannot actually cherry-pick \textit{some} of the good advices from EOL leaving the villain oblivious to the merits of the others. For example, consider this: "\textit{One of my advisors will be an average five-year-old child. Any flaws in my plan that he is able to spot will be corrected before implementation}". Duly implementation of this policy would eliminate the entire category of typical stupid mistakes by simply raising the general situational awareness and basic foresight. In other words, if you make your villain \textit{smarter} he or she has to become smarter in general, not in a particular narrow aspect of strategic planning.

Finally, most important and hard to overcome problem, often completely overlooked by the commenters on the EOL, is that if an Evil Overlord is ever to follow EOL to the letter at least in parts relevant to the setting, he would very quickly cease to be Evil, be Overlord, or both. Simply because the genre law almost mandates explanation of inherent evil in the Evil as some sort of strategic shortcoming overlooked by the said Evil at the planning stage. That is to say, the reader is led to believe that inherent evil is \textit{non-rational}, caused by some sort of personal resentment, childhood traumatic experience, betrayal of someone closest or suchlike. You rarely see Evil as evil just because it is cool and feels good. Rather it is assumed that some kind of dark secret of internal conflict lies at the root of all evilness and as soon as Evil begin to think and act rationally by virtue of reading EOL and taking it seriously, it will somehow "automagically" turn to a path of eventual redeeming.

And here goes my attempt to overcome all of the aforementioned problems in a way that is both realistic and entertaining. Enjoy!

This is the version compiled on \today.

\section*{Licensing}

\pdforepub{
	
\doclicenseThis

\hspace*{\fill} \qrcode[nolink,level=L,height=3cm]{\doclicenseURL}

}{\doclicenseLongText}